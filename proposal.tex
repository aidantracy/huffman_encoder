\documentclass{article}

\usepackage[english]{babel}

\usepackage[letterpaper,top=2cm,bottom=2cm,left=3cm,right=3cm,marginparwidth=1.75cm]{geometry}

\usepackage{amsmath}
\usepackage{graphicx}
\usepackage[colorlinks=true, allcolors=blue]{hyperref}

\title{Zig Proposal}
\author{Michael Dunn, Aidan Tracy}

\begin{document}
\maketitle

\section{Introduction}

Our motivation to use Zig was founded in a desire to learn a relatively new lower-level programming language. Zig has been gaining popularity over the years due to the C-like nature of the language with modern features included in the language.

\section{Language Overview Proposal}

We plan to teach the class about Zig through an:
\begin{itemize}
    \item Overview of language
    \item Modern features and
    \item Detailed walkthrough of a program written in Zig
\end{itemize}

\section{Programming Assignment Proposal}
Since Zig is considered a low-level language meant to compete with Zig, we decided to design and build a simple Huffman encoder. Building an encoder in Zig will allow us to learn about how to build CLI applications in Zig, learn how an encoder works, and build the necessary data structures surrounding it.

\subsection{Evaluation}
We will evaluate our project by writing unit tests to test blocks of code. This will provide confidence for our integration tests that will test for the integration of various units of code. Further evaluation will include manually running the application and comparing it to another Huffman encoding application to test for correctness.

\subsection{Project Management}

Mike and Aidan will work together to design and build a simple Huffman encoder. As we learn about this language, we will note the modern features that Zig has compared to C and create the presentation for that.

\section{Presentation Proposal}
The final group presentation will include a simple PowerPoint walkthrough of the modern language features of Zig as well as basic operations compared to C.

Following a quick overview of the language, we will move into our project and demonstrate it to the class.

\end{document}